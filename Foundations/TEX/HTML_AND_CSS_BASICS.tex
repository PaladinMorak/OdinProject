\chapter{HTML and CSS Basics}
This chapter mainly covers the \href{https://www.freecodecamp.org}{freeCodeCamp} tutorials around
Responsive Web Design.

\section{Basic CSS}
\subsection{How to Select which elements to style}
\subsubsection*{Selectors}
Selectors can be used to apply CSS styling to all elements of a specific type. This is done through
writing the ELEMENT\_TYPE followed by a list of styling options encased in curly brackets.
\begin{lstlisting}[language=HTML, caption="CSS Element type styling example",captionpos=b,frame=tb]
ELEMENT_TYPE {
    STYLING
}
\end{lstlisting}

\subsubsection*{Classes}
CSS Classes are defined through adding a dot \(.\) in front of CLASS\_NAME. They can be used for styling
several different element types the same way. They are selected by adding the "class" attribute to
an HTML element.
\begin{lstlisting}[language=HTML, caption="CSS Classes Example",captionpos=b,frame=tb]
.CLASS_NAME{
    STYLING
} 

<ELEMENT class="CLASS_NAME"></ELEMENT>
\end{lstlisting}

